\documentclass[sigplan,screen,anonymous,review]{acmart}\settopmatter{printfolios=true,printccs=false,printacmref=false}

% TODO: take a look into the program in which all the 250 sequences were better.

\bibliographystyle{ACM-Reference-Format}

% \citestyle{acmauthoryear}   %% For author/year citations

\usepackage{amssymb}
\usepackage{amsmath}
\usepackage{graphicx}
\usepackage{lineno}
\usepackage{url}
\usepackage{soul}
\usepackage{microtype}

\def\qed{\unskip\kern 10pt{\unitlength1pt\linethickness{.4pt}\framebox(6,6){}}}

\newcommand{\Hapi}{\mbox{\textsc{Hapi}}}

\setlength{\belowcaptionskip}{-7pt}

\soulregister\cite7

\newcommand{\fer}[1]{\textcolor{green}{#1}}
\newcommand{\vin}[1]{\textcolor{blue}{#1}}
\newcommand{\alx}[1]{\textcolor{red}{#1}}


\begin{document}

\title[Hapi: A Domain Specific Language for the Specification of Access Policies]
{Hapi: A Domain Specific Language for the Specification of Access Policies}
% \subtitle{Subtitle Text, if any}

\author{Vin\'{i}cius Juli\~{a}o}
\orcid{0000-0002-8588-8197}
\affiliation{
  \institution{UFMG}
  \streetaddress{Avenida Antônio Carlos, 6627}
  \city{Belo Horizonte}
  \state{Minas Gerais}
  \postcode{31.270-213}
  \country{Brazil}
}
\email{vinicius@dcc.ufmg.br.br}

\author{Alexander Holmquist}
\orcid{0000-0002-8588-8197}
\affiliation{
  \institution{UFMG}
  \streetaddress{Avenida Antônio Carlos, 6627}
  \city{Belo Horizonte}
  \state{Minas Gerais}
  \postcode{31.270-213}
  \country{Brazil}
}
\email{alexander@dcc.ufmg.br.br}

\author{Fernando M. Quint\~{a}o Pereira}
\orcid{nnnn-nnnn-nnnn-nnnn}
\affiliation{
%  \department{DCC}
  \institution{UFMG}
  \streetaddress{Avenida Antônio Carlos, 6627}
  \city{Belo Horizonte}
%  \state{Minas Gerais}
  \postcode{31.270-213}
  \country{Brazil}
}
\email{fernando@dcc.ufmg.br}

%\thanks{Thank people here}                %% \thanks is optional

\begin{abstract}
Hapi (short for Hierarchical Access Policy Implementation), is a specification
language that lets users define access policies.
In a distributed system, an access policy defines the actions that each actor
can perform on the available resources. 
Such policies are important because they ensure the preservation of the
integrity and the privacy of personal information that users might store
into remotely accessed databases.

\fer{Finish the abstract.}
\end{abstract}

%% 2012 ACM Computing Classification System (CSS) concepts
%% Generate at 'http://dl.acm.org/ccs/ccs.cfm'.
\begin{CCSXML}
 <ccs2012>
 <concept>
 <concept_id>10011007.10011006.10011041.10011048</concept_id>
 <concept_desc>Software and its engineering~Runtime environments</concept_desc>
 <concept_significance>500</concept_significance>
 </concept>
 <concept>
 <concept_id>10011007.10011006.10011041</concept_id>
 <concept_desc>Software and its engineering~Compilers</concept_desc>
 <concept_significance>500</concept_significance>
 <concept>
 <concept_id>10011007.10011006.10011072</concept_id>
 <concept_desc>Software and its engineering~Software libraries and repositories</concept_desc>
 <concept_significance>300</concept_significance>
 </concept>
 </ccs2012>
\end{CCSXML}

\ccsdesc[500]{Software and its engineering~Runtime environments}
\ccsdesc[500]{Software and its engineering~Compilers}
\ccsdesc[300]{Software and its engineering~Software libraries and repositories}

\keywords{Benchmark, Repository, Synthesis, Training}  %% \keywords is optional

\maketitle

%\renewcommand{\shortauthors}{Marcos Siraichi, Vin\'{i}cius Santos,
%Sylvain Collange, Fernando Pereira}

\section{Introduction}
\label{sec:intro}

\fer{Context: explain what is access policies; where they are used; add at least five references.}
% References of access policies: YaPPL (for IoT), 

Access policies are a set of rules whose a system uses to allow or deny some
operation.
Immutable rules can also be called static access policies, as login routine,
where the rule is having the tuple $(username, password)$ stored in a database.
But there are systems that need dynamic access policies, in which there is a
way to specify and change such rules.
An example of the second case, is the IoT applications in which it must be
possible to allow or deny a resource of getting information from a sensor in
the network.
For that example, YaPPL\cite{} can be a solution, since that defines a set of
keywords in a specific JSON format.
Using this language it is possible define a rule for an \textit{utilizer}
with a \textit{purpose} that get some information by a \textit{transformation}.
Due to the limited quantity of keywords, the power of that language is also
limited, although it specifies policies dynamically.

% Dar um outro exemplo de linguagem que também possui a limitação de não possuir
% uma declaração dos dados utilizados.

% A linguagem (que eu estudei uma vez e esqueci o nome) permite o uso de tipos
% primitivos de dados, e isso gera maior flexibilidade, assim como Hapi


% This paper is focused in this second class, by introducing a DSL
% (\textit{Domain Specific Language}) capable of expressing a set of rules and
% delivery them for a system in a useful structure.

\fer{Problem: explain why hierarchical access is important. Explain that exists legalease, but that is has never been fully defined.}

\fer{Solution: introduce our implementation of Hapi.}

\fer{Results: describe the toosl that we currently have.}

\section{Examples}
\label{sec:examples}

\fer{We need two or three examples that illustrate how Hapi is cool!}

\fer{We need at least one example that illustrates how Legalease is poorly defined.}

\section{Language Specification}
\label{sec:spec}

\subsection{Syntax}
\label{sub:syntax}

\subsection{Semantics}
\label{sub:semantics}

\section{Tools}
\label{sec:tools}

\subsection{Parser and Translators}
\label{sub:parser}

\subsection{The Graphical User Interface}
\label{sub:gui}

\section{Related Work}
\label{sec:rw}

\section{Conclusion}
\label{sec:conc}

\bibliography{references}

\end{document}
